% Options for packages loaded elsewhere
\PassOptionsToPackage{unicode}{hyperref}
\PassOptionsToPackage{hyphens}{url}
\PassOptionsToPackage{dvipsnames,svgnames,x11names}{xcolor}
%
\documentclass[
  12pt,
]{article}
\usepackage{amsmath,amssymb}
\usepackage{iftex}
\ifPDFTeX
  \usepackage[T1]{fontenc}
  \usepackage[utf8]{inputenc}
  \usepackage{textcomp} % provide euro and other symbols
\else % if luatex or xetex
  \usepackage{unicode-math} % this also loads fontspec
  \defaultfontfeatures{Scale=MatchLowercase}
  \defaultfontfeatures[\rmfamily]{Ligatures=TeX,Scale=1}
\fi
\usepackage{lmodern}
\ifPDFTeX\else
  % xetex/luatex font selection
  \setmainfont[]{Times New Roman}
\fi
% Use upquote if available, for straight quotes in verbatim environments
\IfFileExists{upquote.sty}{\usepackage{upquote}}{}
\IfFileExists{microtype.sty}{% use microtype if available
  \usepackage[]{microtype}
  \UseMicrotypeSet[protrusion]{basicmath} % disable protrusion for tt fonts
}{}
\makeatletter
\@ifundefined{KOMAClassName}{% if non-KOMA class
  \IfFileExists{parskip.sty}{%
    \usepackage{parskip}
  }{% else
    \setlength{\parindent}{0pt}
    \setlength{\parskip}{6pt plus 2pt minus 1pt}}
}{% if KOMA class
  \KOMAoptions{parskip=half}}
\makeatother
\usepackage{xcolor}
\usepackage[margin=1in]{geometry}
\usepackage{color}
\usepackage{fancyvrb}
\newcommand{\VerbBar}{|}
\newcommand{\VERB}{\Verb[commandchars=\\\{\}]}
\DefineVerbatimEnvironment{Highlighting}{Verbatim}{commandchars=\\\{\}}
% Add ',fontsize=\small' for more characters per line
\usepackage{framed}
\definecolor{shadecolor}{RGB}{248,248,248}
\newenvironment{Shaded}{\begin{snugshade}}{\end{snugshade}}
\newcommand{\AlertTok}[1]{\textcolor[rgb]{0.94,0.16,0.16}{#1}}
\newcommand{\AnnotationTok}[1]{\textcolor[rgb]{0.56,0.35,0.01}{\textbf{\textit{#1}}}}
\newcommand{\AttributeTok}[1]{\textcolor[rgb]{0.13,0.29,0.53}{#1}}
\newcommand{\BaseNTok}[1]{\textcolor[rgb]{0.00,0.00,0.81}{#1}}
\newcommand{\BuiltInTok}[1]{#1}
\newcommand{\CharTok}[1]{\textcolor[rgb]{0.31,0.60,0.02}{#1}}
\newcommand{\CommentTok}[1]{\textcolor[rgb]{0.56,0.35,0.01}{\textit{#1}}}
\newcommand{\CommentVarTok}[1]{\textcolor[rgb]{0.56,0.35,0.01}{\textbf{\textit{#1}}}}
\newcommand{\ConstantTok}[1]{\textcolor[rgb]{0.56,0.35,0.01}{#1}}
\newcommand{\ControlFlowTok}[1]{\textcolor[rgb]{0.13,0.29,0.53}{\textbf{#1}}}
\newcommand{\DataTypeTok}[1]{\textcolor[rgb]{0.13,0.29,0.53}{#1}}
\newcommand{\DecValTok}[1]{\textcolor[rgb]{0.00,0.00,0.81}{#1}}
\newcommand{\DocumentationTok}[1]{\textcolor[rgb]{0.56,0.35,0.01}{\textbf{\textit{#1}}}}
\newcommand{\ErrorTok}[1]{\textcolor[rgb]{0.64,0.00,0.00}{\textbf{#1}}}
\newcommand{\ExtensionTok}[1]{#1}
\newcommand{\FloatTok}[1]{\textcolor[rgb]{0.00,0.00,0.81}{#1}}
\newcommand{\FunctionTok}[1]{\textcolor[rgb]{0.13,0.29,0.53}{\textbf{#1}}}
\newcommand{\ImportTok}[1]{#1}
\newcommand{\InformationTok}[1]{\textcolor[rgb]{0.56,0.35,0.01}{\textbf{\textit{#1}}}}
\newcommand{\KeywordTok}[1]{\textcolor[rgb]{0.13,0.29,0.53}{\textbf{#1}}}
\newcommand{\NormalTok}[1]{#1}
\newcommand{\OperatorTok}[1]{\textcolor[rgb]{0.81,0.36,0.00}{\textbf{#1}}}
\newcommand{\OtherTok}[1]{\textcolor[rgb]{0.56,0.35,0.01}{#1}}
\newcommand{\PreprocessorTok}[1]{\textcolor[rgb]{0.56,0.35,0.01}{\textit{#1}}}
\newcommand{\RegionMarkerTok}[1]{#1}
\newcommand{\SpecialCharTok}[1]{\textcolor[rgb]{0.81,0.36,0.00}{\textbf{#1}}}
\newcommand{\SpecialStringTok}[1]{\textcolor[rgb]{0.31,0.60,0.02}{#1}}
\newcommand{\StringTok}[1]{\textcolor[rgb]{0.31,0.60,0.02}{#1}}
\newcommand{\VariableTok}[1]{\textcolor[rgb]{0.00,0.00,0.00}{#1}}
\newcommand{\VerbatimStringTok}[1]{\textcolor[rgb]{0.31,0.60,0.02}{#1}}
\newcommand{\WarningTok}[1]{\textcolor[rgb]{0.56,0.35,0.01}{\textbf{\textit{#1}}}}
\usepackage{graphicx}
\makeatletter
\def\maxwidth{\ifdim\Gin@nat@width>\linewidth\linewidth\else\Gin@nat@width\fi}
\def\maxheight{\ifdim\Gin@nat@height>\textheight\textheight\else\Gin@nat@height\fi}
\makeatother
% Scale images if necessary, so that they will not overflow the page
% margins by default, and it is still possible to overwrite the defaults
% using explicit options in \includegraphics[width, height, ...]{}
\setkeys{Gin}{width=\maxwidth,height=\maxheight,keepaspectratio}
% Set default figure placement to htbp
\makeatletter
\def\fps@figure{htbp}
\makeatother
\setlength{\emergencystretch}{3em} % prevent overfull lines
\providecommand{\tightlist}{%
  \setlength{\itemsep}{0pt}\setlength{\parskip}{0pt}}
\setcounter{secnumdepth}{-\maxdimen} % remove section numbering
\usepackage{lastpage}
\usepackage{fancyhdr}
\usepackage{setspace}
\usepackage{float}
\pagestyle{fancy}
\fancyhead[CO, CE]{Jiaqi Bi}
\fancyhead[LE, RO]{Martel - Daily OA Pain}
\fancyfoot[CO, CE]{\thepage \ of \pageref{LastPage}}
\floatplacement{figure}{H}
\ifLuaTeX
  \usepackage{selnolig}  % disable illegal ligatures
\fi
\IfFileExists{bookmark.sty}{\usepackage{bookmark}}{\usepackage{hyperref}}
\IfFileExists{xurl.sty}{\usepackage{xurl}}{} % add URL line breaks if available
\urlstyle{same}
\hypersetup{
  pdftitle={Martel Paper - Daily OA Pain},
  colorlinks=true,
  linkcolor={Maroon},
  filecolor={Maroon},
  citecolor={Blue},
  urlcolor={blue},
  pdfcreator={LaTeX via pandoc}}

\title{Martel Paper - Daily OA Pain}
\author{}
\date{\vspace{-2.5em}2024-03-18}

\begin{document}
\maketitle

\hypertarget{data-wrangling}{%
\subsection{Data Wrangling}\label{data-wrangling}}

\begin{Shaded}
\begin{Highlighting}[]
\DocumentationTok{\#\# Load packages}
\FunctionTok{library}\NormalTok{(tidyverse)}
\FunctionTok{library}\NormalTok{(ggplot2)}
\FunctionTok{library}\NormalTok{(tidyr)}
\FunctionTok{library}\NormalTok{(haven)}\DocumentationTok{\#\# This library provides functions to read sav file into R}
\FunctionTok{library}\NormalTok{(lme4)}
\FunctionTok{library}\NormalTok{(lmerTest)}
\end{Highlighting}
\end{Shaded}

\begin{Shaded}
\begin{Highlighting}[]
\DocumentationTok{\#\# Read data}
\NormalTok{data\_paper }\OtherTok{\textless{}{-}} \FunctionTok{read\_sav}\NormalTok{(}\StringTok{"Dataset; 2024.1.sav"}\NormalTok{)}
\NormalTok{checkdf3 }\OtherTok{\textless{}{-}}\NormalTok{ data\_paper }\SpecialCharTok{|\textgreater{}} 
  \FunctionTok{subset}\NormalTok{(ID }\SpecialCharTok{==} \DecValTok{2072}\NormalTok{) }\SpecialCharTok{|\textgreater{}} 
  \FunctionTok{select}\NormalTok{(}\FunctionTok{c}\NormalTok{(ID, }
\NormalTok{           Level1\_Even\_DateIn, }
\NormalTok{           Level1\_Even\_TimeIn,}
\NormalTok{           Wave\_Day))}

\DocumentationTok{\#\# Delete Weird ID 2072 those weird reporting days}
\NormalTok{data\_paper }\OtherTok{\textless{}{-}}\NormalTok{ data\_paper }\SpecialCharTok{|\textgreater{}}
  \FunctionTok{filter}\NormalTok{(}\SpecialCharTok{!}\NormalTok{(ID }\SpecialCharTok{==} \DecValTok{2072} \SpecialCharTok{\&}\NormalTok{ Wave\_Day }\SpecialCharTok{\textgreater{}=} \DecValTok{7}\NormalTok{))}
\end{Highlighting}
\end{Shaded}

\hypertarget{adjusting-wave-day}{%
\subsubsection{Adjusting Wave Day}\label{adjusting-wave-day}}

\begin{itemize}
  \item $DT_i$ combines $D_i$ and $T_i$: `DateTime` variable
  \item $DT_0$ is the first response `DateTime` for each patient
  \item $W_i$ is the adjusted `Wave\_Day` variable
  \item Add a grace period $G$ for calculating the adjusted $W_i$, in our case $G=6$ hours
  \item Calculate the datetime difference $H_i$ in **hours** from the first response, incorporating the grace period:
  $$
  H_i = DT_i-DT_{i-1}
  $$
  \item Then apply the grace period indicator $I_i$:
  $$
  I_i=
  \begin{cases}
  1 & \text{if } H_{i} \leq 24 + G \\
  \left\lceil\frac{H_{i} - G}{24}\right\rceil & \text{otherwise}
  \end{cases}
  $$
  \item The initial response for `Wave\_Day` is 1, i.e., $W_0=1$, then the adjusted `Wave\_Day` $W_i$ is 
  $$
  W_i=\sum_{i=0}^{i-1} I_i
  $$
\end{itemize}

\begin{Shaded}
\begin{Highlighting}[]
\DocumentationTok{\#\# Consecutive Days {-} Grace Period 6 hours}
\NormalTok{data\_paper }\OtherTok{\textless{}{-}}\NormalTok{ data\_paper }\SpecialCharTok{|\textgreater{}}
  \FunctionTok{mutate}\NormalTok{(}\AttributeTok{Lev1\_DateTimeIn =} \FunctionTok{as.POSIXct}\NormalTok{(}\FunctionTok{strptime}\NormalTok{(}\FunctionTok{paste}\NormalTok{(Level1\_Even\_DateIn, }
\NormalTok{                                                     Level1\_Even\_TimeIn), }
                                          \AttributeTok{format=}\StringTok{"\%Y{-}\%m{-}\%d \%H:\%M:"}\NormalTok{))) }\SpecialCharTok{|\textgreater{}}
  \FunctionTok{arrange}\NormalTok{(ID, Lev1\_DateTimeIn) }\SpecialCharTok{|\textgreater{}}
  \FunctionTok{group\_by}\NormalTok{(ID) }\SpecialCharTok{|\textgreater{}}
  \FunctionTok{mutate}\NormalTok{(}
    \AttributeTok{TimeDiffHours =} \FunctionTok{as.numeric}\NormalTok{(}\FunctionTok{difftime}\NormalTok{(Lev1\_DateTimeIn, }
                                        \FunctionTok{lag}\NormalTok{(Lev1\_DateTimeIn, }
                                            \AttributeTok{default =} \FunctionTok{first}\NormalTok{(Lev1\_DateTimeIn)), }
                                        \AttributeTok{units =} \StringTok{"hours"}\NormalTok{)), }\CommentTok{\# T diff}
    \AttributeTok{WithinGracePeriod =} \FunctionTok{if\_else}\NormalTok{(TimeDiffHours }\SpecialCharTok{\textless{}=} \DecValTok{30}\NormalTok{, }
                                \DecValTok{1}\NormalTok{, }
                                \FunctionTok{ceiling}\NormalTok{((TimeDiffHours }\SpecialCharTok{{-}} \DecValTok{6}\NormalTok{) }\SpecialCharTok{/} \DecValTok{24}\NormalTok{)), }\CommentTok{\# Check grace period}
    \AttributeTok{Wave\_Day\_Adjusted =} \FunctionTok{cumsum}\NormalTok{(WithinGracePeriod) }\CommentTok{\# Adjusted Wave\_Day}
\NormalTok{  ) }\SpecialCharTok{|\textgreater{}}
  \FunctionTok{ungroup}\NormalTok{()}

\DocumentationTok{\#\#\#\#\#\# Check if the above approach is correct \#\#\#\#\#\#}
\NormalTok{checkdf }\OtherTok{\textless{}{-}}\NormalTok{ data\_paper }\SpecialCharTok{|\textgreater{}} \FunctionTok{select}\NormalTok{(}\FunctionTok{c}\NormalTok{(ID, }
\NormalTok{                                  Lev1\_DateTimeIn, }
\NormalTok{                                  TimeDiffHours, }
\NormalTok{                                  WithinGracePeriod, }
\NormalTok{                                  Wave\_Day\_Adjusted,}
\NormalTok{                                  Baseline\_Demog\_BMI))}
\NormalTok{checkdf2 }\OtherTok{\textless{}{-}}\NormalTok{ checkdf }\SpecialCharTok{|\textgreater{}} \FunctionTok{subset}\NormalTok{(ID }\SpecialCharTok{==} \DecValTok{2072}\NormalTok{) }\CommentTok{\# Weird ID 2072}
\FunctionTok{max}\NormalTok{(checkdf}\SpecialCharTok{$}\NormalTok{Wave\_Day\_Adjusted, }\AttributeTok{na.rm =} \ConstantTok{TRUE}\NormalTok{) }
\end{Highlighting}
\end{Shaded}

\begin{verbatim}
## [1] 21
\end{verbatim}

\begin{Shaded}
\begin{Highlighting}[]
\DocumentationTok{\#\# Fill in the gap of Wave\_Day}
\NormalTok{data\_paper2 }\OtherTok{\textless{}{-}}\NormalTok{ data\_paper }\SpecialCharTok{|\textgreater{}}
  \FunctionTok{group\_by}\NormalTok{(ID) }\SpecialCharTok{|\textgreater{}}
  \FunctionTok{complete}\NormalTok{(}\AttributeTok{Wave\_Day =} \DecValTok{1}\SpecialCharTok{:}\DecValTok{14}\NormalTok{) }\SpecialCharTok{|\textgreater{}}
  \FunctionTok{ungroup}\NormalTok{()}
\end{Highlighting}
\end{Shaded}

\begin{Shaded}
\begin{Highlighting}[]
\DocumentationTok{\#\# Check the aberrant values}
\FunctionTok{summary}\NormalTok{(data\_paper2}\SpecialCharTok{$}\NormalTok{IndexLev1\_NegativeAffect\_Total) }\CommentTok{\# Lev 1 Negative Affect }
\end{Highlighting}
\end{Shaded}

\begin{verbatim}
##    Min. 1st Qu.  Median    Mean 3rd Qu.    Max.    NA's 
##   0.000   8.333  20.667  25.189  37.000  97.333    1228
\end{verbatim}

\begin{Shaded}
\begin{Highlighting}[]
\FunctionTok{summary}\NormalTok{(data\_paper2}\SpecialCharTok{$}\NormalTok{IndexLev1\_Catastrophizing\_Total) }\CommentTok{\# Lev 1 Catas}
\end{Highlighting}
\end{Shaded}

\begin{verbatim}
##    Min. 1st Qu.  Median    Mean 3rd Qu.    Max.    NA's 
##   0.000   2.667  12.667  23.085  38.000 100.000    1228
\end{verbatim}

\begin{Shaded}
\begin{Highlighting}[]
\FunctionTok{summary}\NormalTok{(data\_paper2}\SpecialCharTok{$}\NormalTok{IndexLev2\_QST\_BaselinePPTh) }\CommentTok{\# Lev 2 PPThs?}
\end{Highlighting}
\end{Shaded}

\begin{verbatim}
##    Min. 1st Qu.  Median    Mean 3rd Qu.    Max.    NA's 
##    67.0   235.5   369.0   395.8   482.5  1200.0    1321
\end{verbatim}

\begin{Shaded}
\begin{Highlighting}[]
\FunctionTok{summary}\NormalTok{(data\_paper2}\SpecialCharTok{$}\NormalTok{IndexLev2\_QST\_TSPAve) }\CommentTok{\# Lev 2 TSP?}
\end{Highlighting}
\end{Shaded}

\begin{verbatim}
##    Min. 1st Qu.  Median    Mean 3rd Qu.    Max.    NA's 
##   -2.50    1.50    7.50   13.97   15.00   94.00    1284
\end{verbatim}

\begin{Shaded}
\begin{Highlighting}[]
\FunctionTok{summary}\NormalTok{(data\_paper2}\SpecialCharTok{$}\NormalTok{IndexLev2\_QST\_CpmTrialAve) }\CommentTok{\# Lev 2 CPM?}
\end{Highlighting}
\end{Shaded}

\begin{verbatim}
##    Min. 1st Qu.  Median    Mean 3rd Qu.    Max.    NA's 
##   37.56  104.00  119.74  123.10  135.13  251.76    1329
\end{verbatim}

\begin{Shaded}
\begin{Highlighting}[]
\FunctionTok{summary}\NormalTok{(data\_paper2}\SpecialCharTok{$}\NormalTok{IndexLev1\_PainAverage) }\CommentTok{\# Lev 1 Pain?}
\end{Highlighting}
\end{Shaded}

\begin{verbatim}
##    Min. 1st Qu.  Median    Mean 3rd Qu.    Max.    NA's 
##    0.00   24.00   38.00   41.73   59.75  100.00    1234
\end{verbatim}

\begin{Shaded}
\begin{Highlighting}[]
\DocumentationTok{\#\#\#\#\#\# Check if lagged value is correct \#\#\#\#\#\#}
\NormalTok{data\_paper2\_check }\OtherTok{\textless{}{-}}\NormalTok{ data\_paper2 }\SpecialCharTok{|\textgreater{}}
  \FunctionTok{select}\NormalTok{(}\FunctionTok{c}\NormalTok{(ID, Wave\_Day\_Adjusted, IndexLev1\_PainAverage, IndexLev1\_PainAverage\_Lagged))}
\end{Highlighting}
\end{Shaded}

\[
APE(t_i)=I(PAIN(t_i)-PAIN(t_i-1) \geq 20)
\]

\begin{Shaded}
\begin{Highlighting}[]
\DocumentationTok{\#\# APE index}
\NormalTok{data\_paper2 }\OtherTok{\textless{}{-}}\NormalTok{ data\_paper2 }\SpecialCharTok{|\textgreater{}}
  \FunctionTok{group\_by}\NormalTok{(ID) }\SpecialCharTok{|\textgreater{}}
  \FunctionTok{mutate}\NormalTok{(}\AttributeTok{APE =} \FunctionTok{ifelse}\NormalTok{(IndexLev1\_PainAverage }\SpecialCharTok{{-}}\NormalTok{ IndexLev1\_PainAverage\_Lagged }\SpecialCharTok{\textgreater{}=} \DecValTok{20}\NormalTok{, }\DecValTok{1}\NormalTok{, }\DecValTok{0}\NormalTok{))}

\DocumentationTok{\#\#\#\#\#\# Check if it is correctly coded \#\#\#\#\#\#}
\NormalTok{data\_paper2\_check }\OtherTok{\textless{}{-}}\NormalTok{ data\_paper2 }\SpecialCharTok{|\textgreater{}}
  \FunctionTok{select}\NormalTok{(}\FunctionTok{c}\NormalTok{(ID, Wave\_Day\_Adjusted, }
\NormalTok{           IndexLev1\_PainAverage, }
\NormalTok{           IndexLev1\_PainAverage\_Lagged, }
\NormalTok{           APE))}
\end{Highlighting}
\end{Shaded}

For calculating the RPE based on the within person mean, define the
indicator that the pain is above the average pain for person \(i\) on
day \(t_i\). Note that \(n_{t_i}=\max{t_i}\) for patient \(i\).

\[
A(t_i)=I\Big(PAIN(t_i)>\frac{1}{n_{t_i}}\sum_{t_i=1}^{n_{t_i}}PAIN(t_i)\Big)
\] Then define the RPE given \(A(t_i)=1\) for person \(i\) on day
\(t_i\).

\[
RPE(t_i)=I\Big(PAIN(t_i)\leq \frac{1}{n_{t_i}}\sum_{t_i=1}^{n_{t_i}}PAIN(t_i)\Big)\times A(t_i-1)
\]

\begin{Shaded}
\begin{Highlighting}[]
\DocumentationTok{\#\# RPE Index using within person mean}
\NormalTok{data\_paper2 }\OtherTok{\textless{}{-}}\NormalTok{ data\_paper2 }\SpecialCharTok{|\textgreater{}}
  \FunctionTok{group\_by}\NormalTok{(ID) }\SpecialCharTok{|\textgreater{}}
  \FunctionTok{mutate}\NormalTok{(}\AttributeTok{AVE =} \FunctionTok{mean}\NormalTok{(IndexLev1\_PainAverage, }\AttributeTok{na.rm =} \ConstantTok{TRUE}\NormalTok{),}
         \AttributeTok{A =} \FunctionTok{ifelse}\NormalTok{(IndexLev1\_PainAverage }\SpecialCharTok{\textgreater{}}\NormalTok{ AVE, }\DecValTok{1}\NormalTok{, }\DecValTok{0}\NormalTok{),}
         \AttributeTok{A\_lag =} \FunctionTok{lag}\NormalTok{(A),}
         \AttributeTok{Lev1\_RPE\_useMean =} \FunctionTok{ifelse}\NormalTok{(A\_lag }\SpecialCharTok{==} \DecValTok{1}\NormalTok{, }
                                   \FunctionTok{ifelse}\NormalTok{(IndexLev1\_PainAverage }\SpecialCharTok{\textless{}=}\NormalTok{ AVE, }\DecValTok{1}\NormalTok{, }\DecValTok{0}\NormalTok{), }\DecValTok{0}\NormalTok{)) }

\DocumentationTok{\#\#\#\#\#\# Check \#\#\#\#\#\#}
\NormalTok{data\_paper2\_check }\OtherTok{\textless{}{-}}\NormalTok{ data\_paper2 }\SpecialCharTok{|\textgreater{}}
  \FunctionTok{select}\NormalTok{(}\FunctionTok{c}\NormalTok{(ID, Wave\_Day\_Adjusted, IndexLev1\_PainAverage, IndexLev1\_PainAverage\_Lagged, }
\NormalTok{           AVE, A, A\_lag, Lev1\_RPE\_useMean))}
\CommentTok{\# 203 RPEs}
\end{Highlighting}
\end{Shaded}

For calculating the RPE based on the APE,

\[
RPE(t_i)=I\Big(PAIN(t_i)\leq \frac{1}{n_{t_i}}\sum_{t_i=1}^{n_{t_i}}PAIN(t_i)\Big)\times APE(t_i-1)
\]

\begin{Shaded}
\begin{Highlighting}[]
\DocumentationTok{\#\# RPE Index using APE}
\NormalTok{data\_paper2 }\OtherTok{\textless{}{-}}\NormalTok{ data\_paper2 }\SpecialCharTok{|\textgreater{}}
  \FunctionTok{group\_by}\NormalTok{(ID) }\SpecialCharTok{|\textgreater{}}
  \FunctionTok{mutate}\NormalTok{(}\AttributeTok{APE\_lag =} \FunctionTok{lag}\NormalTok{(APE), }
         \AttributeTok{Lev1\_RPE\_useAPE =} \FunctionTok{ifelse}\NormalTok{(APE\_lag }\SpecialCharTok{==} \DecValTok{1}\NormalTok{, }
                                   \FunctionTok{ifelse}\NormalTok{(IndexLev1\_PainAverage }\SpecialCharTok{\textless{}=} \DecValTok{20}\NormalTok{, }\DecValTok{1}\NormalTok{, }\DecValTok{0}\NormalTok{), }\DecValTok{0}\NormalTok{)) }

\DocumentationTok{\#\#\#\#\#\# Check \#\#\#\#\#\#}
\NormalTok{data\_paper2\_check }\OtherTok{\textless{}{-}}\NormalTok{ data\_paper2 }\SpecialCharTok{|\textgreater{}}
  \FunctionTok{select}\NormalTok{(}\FunctionTok{c}\NormalTok{(ID, Wave\_Day\_Adjusted, IndexLev1\_PainAverage, IndexLev1\_PainAverage\_Lagged, }
\NormalTok{           APE, APE\_lag, Lev1\_RPE\_useAPE))}
\CommentTok{\# 4 RPEs}
\end{Highlighting}
\end{Shaded}

\hypertarget{analysis-of-ape}{%
\subsection{Analysis of APE}\label{analysis-of-ape}}

\begin{Shaded}
\begin{Highlighting}[]
\DocumentationTok{\#\# Unadjusted APE to Negative Affect}
\NormalTok{model\_APE.NA }\OtherTok{\textless{}{-}} \FunctionTok{glmer}\NormalTok{(APE }\SpecialCharTok{\textasciitilde{}}\NormalTok{ IndexLev1\_NegativeAffect\_Total }\SpecialCharTok{+}\NormalTok{ (}\DecValTok{1}\SpecialCharTok{|}\NormalTok{ID), }\AttributeTok{family =} \FunctionTok{binomial}\NormalTok{(), }\AttributeTok{data =}\NormalTok{ data\_paper2)}
\FunctionTok{summary}\NormalTok{(model\_APE.NA)}
\end{Highlighting}
\end{Shaded}

\begin{verbatim}
## Generalized linear mixed model fit by maximum likelihood (Laplace
##   Approximation) [glmerMod]
##  Family: binomial  ( logit )
## Formula: APE ~ IndexLev1_NegativeAffect_Total + (1 | ID)
##    Data: data_paper2
## 
##      AIC      BIC   logLik deviance df.resid 
##    451.6    465.7   -222.8    445.6      809 
## 
## Scaled residuals: 
##     Min      1Q  Median      3Q     Max 
## -0.5907 -0.2780 -0.2335 -0.2048  3.9737 
## 
## Random effects:
##  Groups Name        Variance Std.Dev.
##  ID     (Intercept) 0.7268   0.8525  
## Number of obs: 812, groups:  ID, 157
## 
## Fixed effects:
##                                 Estimate Std. Error z value Pr(>|z|)    
## (Intercept)                    -3.124537   0.329004  -9.497   <2e-16 ***
## IndexLev1_NegativeAffect_Total  0.014167   0.006951   2.038   0.0415 *  
## ---
## Signif. codes:  0 '***' 0.001 '**' 0.01 '*' 0.05 '.' 0.1 ' ' 1
## 
## Correlation of Fixed Effects:
##             (Intr)
## IndxL1_NA_T -0.732
\end{verbatim}

\begin{Shaded}
\begin{Highlighting}[]
 \DocumentationTok{\#\# Unadjusted APE to Catastrophizing}
\NormalTok{model\_APE.Cata }\OtherTok{\textless{}{-}} \FunctionTok{glmer}\NormalTok{(APE }\SpecialCharTok{\textasciitilde{}}\NormalTok{ IndexLev1\_Catastrophizing\_Total }\SpecialCharTok{+}\NormalTok{ (}\DecValTok{1}\SpecialCharTok{|}\NormalTok{ID), }\AttributeTok{family =} \FunctionTok{binomial}\NormalTok{(), }\AttributeTok{data =}\NormalTok{ data\_paper2)}
\FunctionTok{summary}\NormalTok{(model\_APE.Cata)}
\end{Highlighting}
\end{Shaded}

\begin{verbatim}
## Generalized linear mixed model fit by maximum likelihood (Laplace
##   Approximation) [glmerMod]
##  Family: binomial  ( logit )
## Formula: APE ~ IndexLev1_Catastrophizing_Total + (1 | ID)
##    Data: data_paper2
## 
##      AIC      BIC   logLik deviance df.resid 
##    444.2    458.3   -219.1    438.2      809 
## 
## Scaled residuals: 
##     Min      1Q  Median      3Q     Max 
## -0.6057 -0.2835 -0.2176 -0.1865  3.9271 
## 
## Random effects:
##  Groups Name        Variance Std.Dev.
##  ID     (Intercept) 0.7367   0.8583  
## Number of obs: 812, groups:  ID, 157
## 
## Fixed effects:
##                                  Estimate Std. Error z value Pr(>|z|)    
## (Intercept)                     -3.252303   0.316961 -10.261  < 2e-16 ***
## IndexLev1_Catastrophizing_Total  0.019020   0.005777   3.292 0.000994 ***
## ---
## Signif. codes:  0 '***' 0.001 '**' 0.01 '*' 0.05 '.' 0.1 ' ' 1
## 
## Correlation of Fixed Effects:
##             (Intr)
## IndxLv1_C_T -0.720
\end{verbatim}

\begin{Shaded}
\begin{Highlighting}[]
\DocumentationTok{\#\# Unadjusted APE to PPThs}
\NormalTok{model\_APE.PPThs }\OtherTok{\textless{}{-}} \FunctionTok{glmer}\NormalTok{(APE }\SpecialCharTok{\textasciitilde{}}\NormalTok{ IndexLev2\_QST\_BaselinePPTh }\SpecialCharTok{+}\NormalTok{ (}\DecValTok{1}\SpecialCharTok{|}\NormalTok{ID), }\AttributeTok{family =} \FunctionTok{binomial}\NormalTok{(), }\AttributeTok{data =}\NormalTok{ data\_paper2)}
\end{Highlighting}
\end{Shaded}

\begin{verbatim}
## Warning in checkConv(attr(opt, "derivs"), opt$par, ctrl = control$checkConv, :
## Model failed to converge with max|grad| = 0.00456593 (tol = 0.002, component 1)
\end{verbatim}

\begin{verbatim}
## Warning in checkConv(attr(opt, "derivs"), opt$par, ctrl = control$checkConv, : Model is nearly unidentifiable: very large eigenvalue
##  - Rescale variables?;Model is nearly unidentifiable: large eigenvalue ratio
##  - Rescale variables?
\end{verbatim}

\begin{Shaded}
\begin{Highlighting}[]
\FunctionTok{summary}\NormalTok{(model\_APE.PPThs)}
\end{Highlighting}
\end{Shaded}

\begin{verbatim}
## Generalized linear mixed model fit by maximum likelihood (Laplace
##   Approximation) [glmerMod]
##  Family: binomial  ( logit )
## Formula: APE ~ IndexLev2_QST_BaselinePPTh + (1 | ID)
##    Data: data_paper2
## 
##      AIC      BIC   logLik deviance df.resid 
##    415.4    429.2   -204.7    409.4      726 
## 
## Scaled residuals: 
##     Min      1Q  Median      3Q     Max 
## -0.4954 -0.2973 -0.2563 -0.2168  4.4125 
## 
## Random effects:
##  Groups Name        Variance Std.Dev.
##  ID     (Intercept) 0.49     0.7     
## Number of obs: 729, groups:  ID, 142
## 
## Fixed effects:
##                              Estimate Std. Error z value Pr(>|z|)    
## (Intercept)                -1.7980758  0.3779896  -4.757 1.97e-06 ***
## IndexLev2_QST_BaselinePPTh -0.0021978  0.0009134  -2.406   0.0161 *  
## ---
## Signif. codes:  0 '***' 0.001 '**' 0.01 '*' 0.05 '.' 0.1 ' ' 1
## 
## Correlation of Fixed Effects:
##             (Intr)
## IL2_QST_BPP -0.807
## optimizer (Nelder_Mead) convergence code: 0 (OK)
## Model failed to converge with max|grad| = 0.00456593 (tol = 0.002, component 1)
## Model is nearly unidentifiable: very large eigenvalue
##  - Rescale variables?
## Model is nearly unidentifiable: large eigenvalue ratio
##  - Rescale variables?
\end{verbatim}

\begin{Shaded}
\begin{Highlighting}[]
\DocumentationTok{\#\# Unadjusted APE to TSP}
\NormalTok{model\_APE.TSP }\OtherTok{\textless{}{-}} \FunctionTok{glmer}\NormalTok{(APE }\SpecialCharTok{\textasciitilde{}}\NormalTok{ IndexLev2\_QST\_TSPAve }\SpecialCharTok{+}\NormalTok{ (}\DecValTok{1}\SpecialCharTok{|}\NormalTok{ID), }\AttributeTok{family =} \FunctionTok{binomial}\NormalTok{(), }\AttributeTok{data =}\NormalTok{ data\_paper2)}
\FunctionTok{summary}\NormalTok{(model\_APE.TSP)}
\end{Highlighting}
\end{Shaded}

\begin{verbatim}
## Generalized linear mixed model fit by maximum likelihood (Laplace
##   Approximation) [glmerMod]
##  Family: binomial  ( logit )
## Formula: APE ~ IndexLev2_QST_TSPAve + (1 | ID)
##    Data: data_paper2
## 
##      AIC      BIC   logLik deviance df.resid 
##    433.9    447.8   -213.9    427.9      758 
## 
## Scaled residuals: 
##     Min      1Q  Median      3Q     Max 
## -0.4841 -0.2851 -0.2630 -0.2260  5.6749 
## 
## Random effects:
##  Groups Name        Variance Std.Dev.
##  ID     (Intercept) 0.512    0.7155  
## Number of obs: 761, groups:  ID, 146
## 
## Fixed effects:
##                      Estimate Std. Error z value Pr(>|z|)    
## (Intercept)          -2.40370    0.24326  -9.881   <2e-16 ***
## IndexLev2_QST_TSPAve -0.01805    0.01029  -1.754   0.0794 .  
## ---
## Signif. codes:  0 '***' 0.001 '**' 0.01 '*' 0.05 '.' 0.1 ' ' 1
## 
## Correlation of Fixed Effects:
##             (Intr)
## IL2_QST_TSP -0.450
\end{verbatim}

\begin{Shaded}
\begin{Highlighting}[]
\DocumentationTok{\#\# Unadjusted APE to CPM}
\NormalTok{model\_APE.CPM }\OtherTok{\textless{}{-}} \FunctionTok{glmer}\NormalTok{(APE }\SpecialCharTok{\textasciitilde{}}\NormalTok{ IndexLev2\_QST\_CpmTrialAve }\SpecialCharTok{+}\NormalTok{ (}\DecValTok{1}\SpecialCharTok{|}\NormalTok{ID), }\AttributeTok{family =} \FunctionTok{binomial}\NormalTok{(), }\AttributeTok{data =}\NormalTok{ data\_paper2)}
\end{Highlighting}
\end{Shaded}

\begin{verbatim}
## Warning in checkConv(attr(opt, "derivs"), opt$par, ctrl = control$checkConv, : Model is nearly unidentifiable: very large eigenvalue
##  - Rescale variables?
\end{verbatim}

\begin{Shaded}
\begin{Highlighting}[]
\FunctionTok{summary}\NormalTok{(model\_APE.CPM)}
\end{Highlighting}
\end{Shaded}

\begin{verbatim}
## Generalized linear mixed model fit by maximum likelihood (Laplace
##   Approximation) [glmerMod]
##  Family: binomial  ( logit )
## Formula: APE ~ IndexLev2_QST_CpmTrialAve + (1 | ID)
##    Data: data_paper2
## 
##      AIC      BIC   logLik deviance df.resid 
##    415.4    429.2   -204.7    409.4      719 
## 
## Scaled residuals: 
##     Min      1Q  Median      3Q     Max 
## -0.4942 -0.2957 -0.2365 -0.2284  3.3972 
## 
## Random effects:
##  Groups Name        Variance Std.Dev.
##  ID     (Intercept) 0.7285   0.8536  
## Number of obs: 722, groups:  ID, 141
## 
## Fixed effects:
##                            Estimate Std. Error z value Pr(>|z|)    
## (Intercept)               -2.895586   0.682405  -4.243  2.2e-05 ***
## IndexLev2_QST_CpmTrialAve  0.001580   0.005063   0.312    0.755    
## ---
## Signif. codes:  0 '***' 0.001 '**' 0.01 '*' 0.05 '.' 0.1 ' ' 1
## 
## Correlation of Fixed Effects:
##             (Intr)
## IL2_QST_CTA -0.939
## optimizer (Nelder_Mead) convergence code: 0 (OK)
## Model is nearly unidentifiable: very large eigenvalue
##  - Rescale variables?
\end{verbatim}

\begin{Shaded}
\begin{Highlighting}[]
\DocumentationTok{\#\# Adjusted APE}
\NormalTok{model\_APE }\OtherTok{\textless{}{-}} \FunctionTok{glmer}\NormalTok{(APE }\SpecialCharTok{\textasciitilde{}}\NormalTok{ IndexLev1\_Catastrophizing\_Total }\SpecialCharTok{+}\NormalTok{ IndexLev1\_NegativeAffect\_Total }\SpecialCharTok{+}\NormalTok{ (}\DecValTok{1}\SpecialCharTok{|}\NormalTok{ID), }\AttributeTok{family =} \FunctionTok{binomial}\NormalTok{(), }\AttributeTok{data =}\NormalTok{ data\_paper2)}
\FunctionTok{summary}\NormalTok{(model\_APE)}
\end{Highlighting}
\end{Shaded}

\begin{verbatim}
## Generalized linear mixed model fit by maximum likelihood (Laplace
##   Approximation) [glmerMod]
##  Family: binomial  ( logit )
## Formula: 
## APE ~ IndexLev1_Catastrophizing_Total + IndexLev1_NegativeAffect_Total +  
##     (1 | ID)
##    Data: data_paper2
## 
##      AIC      BIC   logLik deviance df.resid 
##    445.9    464.7   -219.0    437.9      807 
## 
## Scaled residuals: 
##     Min      1Q  Median      3Q     Max 
## -0.6339 -0.2812 -0.2176 -0.1846  3.9902 
## 
## Random effects:
##  Groups Name        Variance Std.Dev.
##  ID     (Intercept) 0.7666   0.8755  
## Number of obs: 811, groups:  ID, 157
## 
## Fixed effects:
##                                  Estimate Std. Error z value Pr(>|z|)    
## (Intercept)                     -3.326941   0.364314  -9.132  < 2e-16 ***
## IndexLev1_Catastrophizing_Total  0.017634   0.006465   2.728  0.00638 ** 
## IndexLev1_NegativeAffect_Total   0.003769   0.008028   0.470  0.63869    
## ---
## Signif. codes:  0 '***' 0.001 '**' 0.01 '*' 0.05 '.' 0.1 ' ' 1
## 
## Correlation of Fixed Effects:
##             (Intr) IL1_C_
## IndxLv1_C_T -0.361       
## IndxL1_NA_T -0.481 -0.434
\end{verbatim}

\begin{Shaded}
\begin{Highlighting}[]
\NormalTok{model\_APE\_2 }\OtherTok{\textless{}{-}} \FunctionTok{glmer}\NormalTok{(APE }\SpecialCharTok{\textasciitilde{}}\NormalTok{ IndexLev2\_QST\_CpmTrialAve }\SpecialCharTok{+}\NormalTok{ IndexLev2\_QST\_TSPAve }\SpecialCharTok{+}\NormalTok{ IndexLev2\_QST\_BaselinePPTh }\SpecialCharTok{+}\NormalTok{ (}\DecValTok{1}\SpecialCharTok{|}\NormalTok{ID), }\AttributeTok{family =} \FunctionTok{binomial}\NormalTok{(), }\AttributeTok{data =}\NormalTok{ data\_paper2)}
\end{Highlighting}
\end{Shaded}

\begin{verbatim}
## Warning in checkConv(attr(opt, "derivs"), opt$par, ctrl = control$checkConv, :
## Model failed to converge with max|grad| = 0.0082828 (tol = 0.002, component 1)
\end{verbatim}

\begin{verbatim}
## Warning in checkConv(attr(opt, "derivs"), opt$par, ctrl = control$checkConv, : Model is nearly unidentifiable: very large eigenvalue
##  - Rescale variables?;Model is nearly unidentifiable: large eigenvalue ratio
##  - Rescale variables?
\end{verbatim}

\begin{Shaded}
\begin{Highlighting}[]
\FunctionTok{summary}\NormalTok{(model\_APE\_2)}
\end{Highlighting}
\end{Shaded}

\begin{verbatim}
## Generalized linear mixed model fit by maximum likelihood (Laplace
##   Approximation) [glmerMod]
##  Family: binomial  ( logit )
## Formula: 
## APE ~ IndexLev2_QST_CpmTrialAve + IndexLev2_QST_TSPAve + IndexLev2_QST_BaselinePPTh +  
##     (1 | ID)
##    Data: data_paper2
## 
##      AIC      BIC   logLik deviance df.resid 
##    395.4    418.1   -192.7    385.4      679 
## 
## Scaled residuals: 
##     Min      1Q  Median      3Q     Max 
## -0.4812 -0.3194 -0.2721 -0.2041  6.7952 
## 
## Random effects:
##  Groups Name        Variance Std.Dev.
##  ID     (Intercept) 0.2546   0.5046  
## Number of obs: 684, groups:  ID, 132
## 
## Fixed effects:
##                              Estimate Std. Error z value Pr(>|z|)   
## (Intercept)                -0.8004088  0.8281574  -0.966  0.33380   
## IndexLev2_QST_CpmTrialAve  -0.0025796  0.0048250  -0.535  0.59290   
## IndexLev2_QST_TSPAve       -0.0250571  0.0106213  -2.359  0.01832 * 
## IndexLev2_QST_BaselinePPTh -0.0028134  0.0009779  -2.877  0.00401 **
## ---
## Signif. codes:  0 '***' 0.001 '**' 0.01 '*' 0.05 '.' 0.1 ' ' 1
## 
## Correlation of Fixed Effects:
##             (Intr) IL2_QST_C IL2_QST_T
## IL2_QST_CTA -0.864                    
## IL2_QST_TSP -0.236  0.030             
## IL2_QST_BPP -0.631  0.275     0.176   
## optimizer (Nelder_Mead) convergence code: 0 (OK)
## Model failed to converge with max|grad| = 0.0082828 (tol = 0.002, component 1)
## Model is nearly unidentifiable: very large eigenvalue
##  - Rescale variables?
## Model is nearly unidentifiable: large eigenvalue ratio
##  - Rescale variables?
\end{verbatim}

\hypertarget{analysis-of-rpe}{%
\subsection{Analysis of RPE}\label{analysis-of-rpe}}

\begin{Shaded}
\begin{Highlighting}[]
\DocumentationTok{\#\# Unadjusted RPE to Negative Affect}
\NormalTok{model\_RPE.NA }\OtherTok{\textless{}{-}} \FunctionTok{glmer}\NormalTok{(Lev1\_RPE\_useMean }\SpecialCharTok{\textasciitilde{}}\NormalTok{ IndexLev1\_NegativeAffect\_Total }\SpecialCharTok{+}\NormalTok{ (}\DecValTok{1}\SpecialCharTok{|}\NormalTok{ID), }\AttributeTok{family =} \FunctionTok{binomial}\NormalTok{(), }\AttributeTok{data =}\NormalTok{ data\_paper2)}
\end{Highlighting}
\end{Shaded}

\begin{verbatim}
## boundary (singular) fit: see help('isSingular')
\end{verbatim}

\begin{Shaded}
\begin{Highlighting}[]
\FunctionTok{summary}\NormalTok{(model\_RPE.NA)}
\end{Highlighting}
\end{Shaded}

\begin{verbatim}
## Generalized linear mixed model fit by maximum likelihood (Laplace
##   Approximation) [glmerMod]
##  Family: binomial  ( logit )
## Formula: Lev1_RPE_useMean ~ IndexLev1_NegativeAffect_Total + (1 | ID)
##    Data: data_paper2
## 
##      AIC      BIC   logLik deviance df.resid 
##    923.8    937.9   -458.9    917.8      814 
## 
## Scaled residuals: 
##     Min      1Q  Median      3Q     Max 
## -0.5986 -0.5882 -0.5715 -0.5178  1.9000 
## 
## Random effects:
##  Groups Name        Variance Std.Dev.
##  ID     (Intercept) 0        0       
## Number of obs: 817, groups:  ID, 157
## 
## Fixed effects:
##                                 Estimate Std. Error z value Pr(>|z|)    
## (Intercept)                    -1.026494   0.127343  -8.061 7.58e-16 ***
## IndexLev1_NegativeAffect_Total -0.002979   0.004024  -0.740    0.459    
## ---
## Signif. codes:  0 '***' 0.001 '**' 0.01 '*' 0.05 '.' 0.1 ' ' 1
## 
## Correlation of Fixed Effects:
##             (Intr)
## IndxL1_NA_T -0.773
## optimizer (Nelder_Mead) convergence code: 0 (OK)
## boundary (singular) fit: see help('isSingular')
\end{verbatim}

\begin{Shaded}
\begin{Highlighting}[]
 \DocumentationTok{\#\# Unadjusted RPE to Catastrophizing}
\NormalTok{model\_RPE.Cata }\OtherTok{\textless{}{-}} \FunctionTok{glmer}\NormalTok{(Lev1\_RPE\_useMean }\SpecialCharTok{\textasciitilde{}}\NormalTok{ IndexLev1\_Catastrophizing\_Total }\SpecialCharTok{+}\NormalTok{ (}\DecValTok{1}\SpecialCharTok{|}\NormalTok{ID), }\AttributeTok{family =} \FunctionTok{binomial}\NormalTok{(), }\AttributeTok{data =}\NormalTok{ data\_paper2)}
\end{Highlighting}
\end{Shaded}

\begin{verbatim}
## boundary (singular) fit: see help('isSingular')
\end{verbatim}

\begin{Shaded}
\begin{Highlighting}[]
\FunctionTok{summary}\NormalTok{(model\_RPE.Cata)}
\end{Highlighting}
\end{Shaded}

\begin{verbatim}
## Generalized linear mixed model fit by maximum likelihood (Laplace
##   Approximation) [glmerMod]
##  Family: binomial  ( logit )
## Formula: Lev1_RPE_useMean ~ IndexLev1_Catastrophizing_Total + (1 | ID)
##    Data: data_paper2
## 
##      AIC      BIC   logLik deviance df.resid 
##    920.2    934.3   -457.1    914.2      814 
## 
## Scaled residuals: 
##     Min      1Q  Median      3Q     Max 
## -0.6210 -0.6084 -0.5641 -0.4431  2.2700 
## 
## Random effects:
##  Groups Name        Variance Std.Dev.
##  ID     (Intercept) 0        0       
## Number of obs: 817, groups:  ID, 157
## 
## Fixed effects:
##                                  Estimate Std. Error z value Pr(>|z|)    
## (Intercept)                     -0.952720   0.107572  -8.857   <2e-16 ***
## IndexLev1_Catastrophizing_Total -0.006868   0.003466  -1.981   0.0475 *  
## ---
## Signif. codes:  0 '***' 0.001 '**' 0.01 '*' 0.05 '.' 0.1 ' ' 1
## 
## Correlation of Fixed Effects:
##             (Intr)
## IndxLv1_C_T -0.658
## optimizer (Nelder_Mead) convergence code: 0 (OK)
## boundary (singular) fit: see help('isSingular')
\end{verbatim}

\begin{Shaded}
\begin{Highlighting}[]
\DocumentationTok{\#\# Unadjusted RPE to PPThs}
\NormalTok{model\_RPE.PPThs }\OtherTok{\textless{}{-}} \FunctionTok{glmer}\NormalTok{(Lev1\_RPE\_useMean }\SpecialCharTok{\textasciitilde{}}\NormalTok{ IndexLev2\_QST\_BaselinePPTh }\SpecialCharTok{+}\NormalTok{ (}\DecValTok{1}\SpecialCharTok{|}\NormalTok{ID), }\AttributeTok{family =} \FunctionTok{binomial}\NormalTok{(), }\AttributeTok{data =}\NormalTok{ data\_paper2)}
\end{Highlighting}
\end{Shaded}

\begin{verbatim}
## boundary (singular) fit: see help('isSingular')
\end{verbatim}

\begin{Shaded}
\begin{Highlighting}[]
\FunctionTok{summary}\NormalTok{(model\_RPE.PPThs)}
\end{Highlighting}
\end{Shaded}

\begin{verbatim}
## Generalized linear mixed model fit by maximum likelihood (Laplace
##   Approximation) [glmerMod]
##  Family: binomial  ( logit )
## Formula: Lev1_RPE_useMean ~ IndexLev2_QST_BaselinePPTh + (1 | ID)
##    Data: data_paper2
## 
##      AIC      BIC   logLik deviance df.resid 
##    837.0    850.8   -415.5    831.0      731 
## 
## Scaled residuals: 
##     Min      1Q  Median      3Q     Max 
## -0.6444 -0.6050 -0.5738  1.5684  2.1812 
## 
## Random effects:
##  Groups Name        Variance Std.Dev.
##  ID     (Intercept) 0        0       
## Number of obs: 734, groups:  ID, 142
## 
## Fixed effects:
##                              Estimate Std. Error z value Pr(>|z|)    
## (Intercept)                -0.8386985  0.1835064  -4.570 4.87e-06 ***
## IndexLev2_QST_BaselinePPTh -0.0006009  0.0004239  -1.418    0.156    
## ---
## Signif. codes:  0 '***' 0.001 '**' 0.01 '*' 0.05 '.' 0.1 ' ' 1
## 
## Correlation of Fixed Effects:
##             (Intr)
## IL2_QST_BPP -0.887
## optimizer (Nelder_Mead) convergence code: 0 (OK)
## boundary (singular) fit: see help('isSingular')
\end{verbatim}

\begin{Shaded}
\begin{Highlighting}[]
\DocumentationTok{\#\# Unadjusted RPE to TSP}
\NormalTok{model\_RPE.TSP }\OtherTok{\textless{}{-}} \FunctionTok{glmer}\NormalTok{(Lev1\_RPE\_useMean }\SpecialCharTok{\textasciitilde{}}\NormalTok{ IndexLev2\_QST\_TSPAve }\SpecialCharTok{+}\NormalTok{ (}\DecValTok{1}\SpecialCharTok{|}\NormalTok{ID), }\AttributeTok{family =} \FunctionTok{binomial}\NormalTok{(), }\AttributeTok{data =}\NormalTok{ data\_paper2)}
\end{Highlighting}
\end{Shaded}

\begin{verbatim}
## boundary (singular) fit: see help('isSingular')
\end{verbatim}

\begin{Shaded}
\begin{Highlighting}[]
\FunctionTok{summary}\NormalTok{(model\_RPE.TSP)}
\end{Highlighting}
\end{Shaded}

\begin{verbatim}
## Generalized linear mixed model fit by maximum likelihood (Laplace
##   Approximation) [glmerMod]
##  Family: binomial  ( logit )
## Formula: Lev1_RPE_useMean ~ IndexLev2_QST_TSPAve + (1 | ID)
##    Data: data_paper2
## 
##      AIC      BIC   logLik deviance df.resid 
##    868.0    881.9   -431.0    862.0      762 
## 
## Scaled residuals: 
##     Min      1Q  Median      3Q     Max 
## -0.5804 -0.5799 -0.5790  1.7229  1.7501 
## 
## Random effects:
##  Groups Name        Variance Std.Dev.
##  ID     (Intercept) 0        0       
## Number of obs: 765, groups:  ID, 146
## 
## Fixed effects:
##                        Estimate Std. Error z value Pr(>|z|)    
## (Intercept)          -1.0888762  0.1040629 -10.464   <2e-16 ***
## IndexLev2_QST_TSPAve -0.0003236  0.0044722  -0.072    0.942    
## ---
## Signif. codes:  0 '***' 0.001 '**' 0.01 '*' 0.05 '.' 0.1 ' ' 1
## 
## Correlation of Fixed Effects:
##             (Intr)
## IL2_QST_TSP -0.598
## optimizer (Nelder_Mead) convergence code: 0 (OK)
## boundary (singular) fit: see help('isSingular')
\end{verbatim}

\begin{Shaded}
\begin{Highlighting}[]
\DocumentationTok{\#\# Unadjusted RPE to CPM}
\NormalTok{model\_RPE.CPM }\OtherTok{\textless{}{-}} \FunctionTok{glmer}\NormalTok{(Lev1\_RPE\_useMean }\SpecialCharTok{\textasciitilde{}}\NormalTok{ IndexLev2\_QST\_CpmTrialAve }\SpecialCharTok{+}\NormalTok{ (}\DecValTok{1}\SpecialCharTok{|}\NormalTok{ID), }\AttributeTok{family =} \FunctionTok{binomial}\NormalTok{(), }\AttributeTok{data =}\NormalTok{ data\_paper2)}
\end{Highlighting}
\end{Shaded}

\begin{verbatim}
## boundary (singular) fit: see help('isSingular')
\end{verbatim}

\begin{Shaded}
\begin{Highlighting}[]
\FunctionTok{summary}\NormalTok{(model\_RPE.CPM)}
\end{Highlighting}
\end{Shaded}

\begin{verbatim}
## Generalized linear mixed model fit by maximum likelihood (Laplace
##   Approximation) [glmerMod]
##  Family: binomial  ( logit )
## Formula: Lev1_RPE_useMean ~ IndexLev2_QST_CpmTrialAve + (1 | ID)
##    Data: data_paper2
## 
##      AIC      BIC   logLik deviance df.resid 
##    828.3    842.1   -411.1    822.3      724 
## 
## Scaled residuals: 
##     Min      1Q  Median      3Q     Max 
## -0.6344 -0.5849 -0.5771  1.5998  1.8201 
## 
## Random effects:
##  Groups Name        Variance  Std.Dev. 
##  ID     (Intercept) 3.286e-16 1.813e-08
## Number of obs: 727, groups:  ID, 141
## 
## Fixed effects:
##                            Estimate Std. Error z value Pr(>|z|)    
## (Intercept)               -1.248189   0.345374  -3.614 0.000301 ***
## IndexLev2_QST_CpmTrialAve  0.001342   0.002699   0.497 0.618951    
## ---
## Signif. codes:  0 '***' 0.001 '**' 0.01 '*' 0.05 '.' 0.1 ' ' 1
## 
## Correlation of Fixed Effects:
##             (Intr)
## IL2_QST_CTA -0.969
## optimizer (Nelder_Mead) convergence code: 0 (OK)
## boundary (singular) fit: see help('isSingular')
\end{verbatim}

\begin{Shaded}
\begin{Highlighting}[]
\DocumentationTok{\#\# Adjusted RPE}
\NormalTok{model\_RPE }\OtherTok{\textless{}{-}} \FunctionTok{glmer}\NormalTok{(Lev1\_RPE\_useMean }\SpecialCharTok{\textasciitilde{}}\NormalTok{  IndexLev1\_Catastrophizing\_Total }\SpecialCharTok{+}\NormalTok{ IndexLev1\_NegativeAffect\_Total }\SpecialCharTok{+}\NormalTok{ (}\DecValTok{1}\SpecialCharTok{|}\NormalTok{ID), }\AttributeTok{family =} \FunctionTok{binomial}\NormalTok{(), }\AttributeTok{data =}\NormalTok{ data\_paper2)}
\end{Highlighting}
\end{Shaded}

\begin{verbatim}
## boundary (singular) fit: see help('isSingular')
\end{verbatim}

\begin{Shaded}
\begin{Highlighting}[]
\FunctionTok{summary}\NormalTok{(model\_RPE)}
\end{Highlighting}
\end{Shaded}

\begin{verbatim}
## Generalized linear mixed model fit by maximum likelihood (Laplace
##   Approximation) [glmerMod]
##  Family: binomial  ( logit )
## Formula: 
## Lev1_RPE_useMean ~ IndexLev1_Catastrophizing_Total + IndexLev1_NegativeAffect_Total +  
##     (1 | ID)
##    Data: data_paper2
## 
##      AIC      BIC   logLik deviance df.resid 
##    919.5    938.4   -455.8    911.5      812 
## 
## Scaled residuals: 
##     Min      1Q  Median      3Q     Max 
## -0.6507 -0.6074 -0.5641 -0.4285  2.3095 
## 
## Random effects:
##  Groups Name        Variance Std.Dev.
##  ID     (Intercept) 0        0       
## Number of obs: 816, groups:  ID, 157
## 
## Fixed effects:
##                                  Estimate Std. Error z value Pr(>|z|)    
## (Intercept)                     -0.985040   0.130399  -7.554 4.22e-14 ***
## IndexLev1_Catastrophizing_Total -0.007347   0.004012  -1.831   0.0671 .  
## IndexLev1_NegativeAffect_Total   0.001537   0.004684   0.328   0.7428    
## ---
## Signif. codes:  0 '***' 0.001 '**' 0.01 '*' 0.05 '.' 0.1 ' ' 1
## 
## Correlation of Fixed Effects:
##             (Intr) IL1_C_
## IndxLv1_C_T -0.189       
## IndxL1_NA_T -0.562 -0.502
## optimizer (Nelder_Mead) convergence code: 0 (OK)
## boundary (singular) fit: see help('isSingular')
\end{verbatim}

\begin{Shaded}
\begin{Highlighting}[]
\NormalTok{model\_RPE\_2 }\OtherTok{\textless{}{-}} \FunctionTok{glmer}\NormalTok{(Lev1\_RPE\_useMean }\SpecialCharTok{\textasciitilde{}}\NormalTok{ IndexLev2\_QST\_CpmTrialAve }\SpecialCharTok{+}\NormalTok{ IndexLev2\_QST\_TSPAve }\SpecialCharTok{+}\NormalTok{ IndexLev2\_QST\_BaselinePPTh }\SpecialCharTok{+}\NormalTok{ (}\DecValTok{1}\SpecialCharTok{|}\NormalTok{ID), }\AttributeTok{family =} \FunctionTok{binomial}\NormalTok{(), }\AttributeTok{data =}\NormalTok{ data\_paper2)}
\end{Highlighting}
\end{Shaded}

\begin{verbatim}
## boundary (singular) fit: see help('isSingular')
\end{verbatim}

\begin{Shaded}
\begin{Highlighting}[]
\FunctionTok{summary}\NormalTok{(model\_RPE\_2)}
\end{Highlighting}
\end{Shaded}

\begin{verbatim}
## Generalized linear mixed model fit by maximum likelihood (Laplace
##   Approximation) [glmerMod]
##  Family: binomial  ( logit )
## Formula: Lev1_RPE_useMean ~ IndexLev2_QST_CpmTrialAve + IndexLev2_QST_TSPAve +  
##     IndexLev2_QST_BaselinePPTh + (1 | ID)
##    Data: data_paper2
## 
##      AIC      BIC   logLik deviance df.resid 
##    786.9    809.6   -388.4    776.9      683 
## 
## Scaled residuals: 
##     Min      1Q  Median      3Q     Max 
## -0.6359 -0.5949 -0.5740  1.5799  2.0883 
## 
## Random effects:
##  Groups Name        Variance Std.Dev.
##  ID     (Intercept) 0        0       
## Number of obs: 688, groups:  ID, 132
## 
## Fixed effects:
##                              Estimate Std. Error z value Pr(>|z|)  
## (Intercept)                -0.8913559  0.4717127  -1.890   0.0588 .
## IndexLev2_QST_CpmTrialAve   0.0001801  0.0028856   0.062   0.9502  
## IndexLev2_QST_TSPAve       -0.0010464  0.0046924  -0.223   0.8235  
## IndexLev2_QST_BaselinePPTh -0.0005068  0.0004886  -1.037   0.2996  
## ---
## Signif. codes:  0 '***' 0.001 '**' 0.01 '*' 0.05 '.' 0.1 ' ' 1
## 
## Correlation of Fixed Effects:
##             (Intr) IL2_QST_C IL2_QST_T
## IL2_QST_CTA -0.878                    
## IL2_QST_TSP -0.281  0.055             
## IL2_QST_BPP -0.645  0.278     0.227   
## optimizer (Nelder_Mead) convergence code: 0 (OK)
## boundary (singular) fit: see help('isSingular')
\end{verbatim}

\end{document}
