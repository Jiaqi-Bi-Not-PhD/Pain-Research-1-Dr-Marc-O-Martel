% Options for packages loaded elsewhere
\PassOptionsToPackage{unicode}{hyperref}
\PassOptionsToPackage{hyphens}{url}
\PassOptionsToPackage{dvipsnames,svgnames,x11names}{xcolor}
%
\documentclass[
  12pt,
]{article}
\usepackage{amsmath,amssymb}
\usepackage{iftex}
\ifPDFTeX
  \usepackage[T1]{fontenc}
  \usepackage[utf8]{inputenc}
  \usepackage{textcomp} % provide euro and other symbols
\else % if luatex or xetex
  \usepackage{unicode-math} % this also loads fontspec
  \defaultfontfeatures{Scale=MatchLowercase}
  \defaultfontfeatures[\rmfamily]{Ligatures=TeX,Scale=1}
\fi
\usepackage{lmodern}
\ifPDFTeX\else
  % xetex/luatex font selection
  \setmainfont[]{Times New Roman}
\fi
% Use upquote if available, for straight quotes in verbatim environments
\IfFileExists{upquote.sty}{\usepackage{upquote}}{}
\IfFileExists{microtype.sty}{% use microtype if available
  \usepackage[]{microtype}
  \UseMicrotypeSet[protrusion]{basicmath} % disable protrusion for tt fonts
}{}
\makeatletter
\@ifundefined{KOMAClassName}{% if non-KOMA class
  \IfFileExists{parskip.sty}{%
    \usepackage{parskip}
  }{% else
    \setlength{\parindent}{0pt}
    \setlength{\parskip}{6pt plus 2pt minus 1pt}}
}{% if KOMA class
  \KOMAoptions{parskip=half}}
\makeatother
\usepackage{xcolor}
\usepackage[margin=1in]{geometry}
\usepackage{color}
\usepackage{fancyvrb}
\newcommand{\VerbBar}{|}
\newcommand{\VERB}{\Verb[commandchars=\\\{\}]}
\DefineVerbatimEnvironment{Highlighting}{Verbatim}{commandchars=\\\{\}}
% Add ',fontsize=\small' for more characters per line
\usepackage{framed}
\definecolor{shadecolor}{RGB}{248,248,248}
\newenvironment{Shaded}{\begin{snugshade}}{\end{snugshade}}
\newcommand{\AlertTok}[1]{\textcolor[rgb]{0.94,0.16,0.16}{#1}}
\newcommand{\AnnotationTok}[1]{\textcolor[rgb]{0.56,0.35,0.01}{\textbf{\textit{#1}}}}
\newcommand{\AttributeTok}[1]{\textcolor[rgb]{0.13,0.29,0.53}{#1}}
\newcommand{\BaseNTok}[1]{\textcolor[rgb]{0.00,0.00,0.81}{#1}}
\newcommand{\BuiltInTok}[1]{#1}
\newcommand{\CharTok}[1]{\textcolor[rgb]{0.31,0.60,0.02}{#1}}
\newcommand{\CommentTok}[1]{\textcolor[rgb]{0.56,0.35,0.01}{\textit{#1}}}
\newcommand{\CommentVarTok}[1]{\textcolor[rgb]{0.56,0.35,0.01}{\textbf{\textit{#1}}}}
\newcommand{\ConstantTok}[1]{\textcolor[rgb]{0.56,0.35,0.01}{#1}}
\newcommand{\ControlFlowTok}[1]{\textcolor[rgb]{0.13,0.29,0.53}{\textbf{#1}}}
\newcommand{\DataTypeTok}[1]{\textcolor[rgb]{0.13,0.29,0.53}{#1}}
\newcommand{\DecValTok}[1]{\textcolor[rgb]{0.00,0.00,0.81}{#1}}
\newcommand{\DocumentationTok}[1]{\textcolor[rgb]{0.56,0.35,0.01}{\textbf{\textit{#1}}}}
\newcommand{\ErrorTok}[1]{\textcolor[rgb]{0.64,0.00,0.00}{\textbf{#1}}}
\newcommand{\ExtensionTok}[1]{#1}
\newcommand{\FloatTok}[1]{\textcolor[rgb]{0.00,0.00,0.81}{#1}}
\newcommand{\FunctionTok}[1]{\textcolor[rgb]{0.13,0.29,0.53}{\textbf{#1}}}
\newcommand{\ImportTok}[1]{#1}
\newcommand{\InformationTok}[1]{\textcolor[rgb]{0.56,0.35,0.01}{\textbf{\textit{#1}}}}
\newcommand{\KeywordTok}[1]{\textcolor[rgb]{0.13,0.29,0.53}{\textbf{#1}}}
\newcommand{\NormalTok}[1]{#1}
\newcommand{\OperatorTok}[1]{\textcolor[rgb]{0.81,0.36,0.00}{\textbf{#1}}}
\newcommand{\OtherTok}[1]{\textcolor[rgb]{0.56,0.35,0.01}{#1}}
\newcommand{\PreprocessorTok}[1]{\textcolor[rgb]{0.56,0.35,0.01}{\textit{#1}}}
\newcommand{\RegionMarkerTok}[1]{#1}
\newcommand{\SpecialCharTok}[1]{\textcolor[rgb]{0.81,0.36,0.00}{\textbf{#1}}}
\newcommand{\SpecialStringTok}[1]{\textcolor[rgb]{0.31,0.60,0.02}{#1}}
\newcommand{\StringTok}[1]{\textcolor[rgb]{0.31,0.60,0.02}{#1}}
\newcommand{\VariableTok}[1]{\textcolor[rgb]{0.00,0.00,0.00}{#1}}
\newcommand{\VerbatimStringTok}[1]{\textcolor[rgb]{0.31,0.60,0.02}{#1}}
\newcommand{\WarningTok}[1]{\textcolor[rgb]{0.56,0.35,0.01}{\textbf{\textit{#1}}}}
\usepackage{graphicx}
\makeatletter
\def\maxwidth{\ifdim\Gin@nat@width>\linewidth\linewidth\else\Gin@nat@width\fi}
\def\maxheight{\ifdim\Gin@nat@height>\textheight\textheight\else\Gin@nat@height\fi}
\makeatother
% Scale images if necessary, so that they will not overflow the page
% margins by default, and it is still possible to overwrite the defaults
% using explicit options in \includegraphics[width, height, ...]{}
\setkeys{Gin}{width=\maxwidth,height=\maxheight,keepaspectratio}
% Set default figure placement to htbp
\makeatletter
\def\fps@figure{htbp}
\makeatother
\setlength{\emergencystretch}{3em} % prevent overfull lines
\providecommand{\tightlist}{%
  \setlength{\itemsep}{0pt}\setlength{\parskip}{0pt}}
\setcounter{secnumdepth}{-\maxdimen} % remove section numbering
\usepackage{lastpage}
\usepackage{fancyhdr}
\usepackage{setspace}
\usepackage{float}
\pagestyle{fancy}
\fancyhead[CO, CE]{Jiaqi Bi}
\fancyhead[LE, RO]{Martel et al. Daily OA Pain}
\fancyfoot[CO, CE]{\thepage \ of \pageref{LastPage}}
\floatplacement{figure}{H}
\ifLuaTeX
  \usepackage{selnolig}  % disable illegal ligatures
\fi
\IfFileExists{bookmark.sty}{\usepackage{bookmark}}{\usepackage{hyperref}}
\IfFileExists{xurl.sty}{\usepackage{xurl}}{} % add URL line breaks if available
\urlstyle{same}
\hypersetup{
  pdftitle={Martel Paper - Daily OA Pain},
  colorlinks=true,
  linkcolor={Maroon},
  filecolor={Maroon},
  citecolor={Blue},
  urlcolor={blue},
  pdfcreator={LaTeX via pandoc}}

\title{Martel Paper - Daily OA Pain}
\author{}
\date{\vspace{-2.5em}2024-03-16}

\begin{document}
\maketitle

\hypertarget{data-wrangling}{%
\subsection{Data Wrangling}\label{data-wrangling}}

\begin{Shaded}
\begin{Highlighting}[]
\DocumentationTok{\#\# Load packages}
\FunctionTok{library}\NormalTok{(tidyverse)}
\FunctionTok{library}\NormalTok{(ggplot2)}
\FunctionTok{library}\NormalTok{(tidyr)}
\FunctionTok{library}\NormalTok{(haven)}\DocumentationTok{\#\# This library provides functions to read sav file into R}
\FunctionTok{library}\NormalTok{(lme4)}
\FunctionTok{library}\NormalTok{(lmerTest)}
\end{Highlighting}
\end{Shaded}

\begin{Shaded}
\begin{Highlighting}[]
\DocumentationTok{\#\# Read data}
\NormalTok{data\_paper }\OtherTok{\textless{}{-}} \FunctionTok{read\_sav}\NormalTok{(}\StringTok{"Dataset; 2024.1.sav"}\NormalTok{)}
\NormalTok{checkdf3 }\OtherTok{\textless{}{-}}\NormalTok{ data\_paper }\SpecialCharTok{|\textgreater{}} 
  \FunctionTok{subset}\NormalTok{(ID }\SpecialCharTok{==} \DecValTok{2072}\NormalTok{) }\SpecialCharTok{|\textgreater{}} 
  \FunctionTok{select}\NormalTok{(}\FunctionTok{c}\NormalTok{(ID, }
\NormalTok{           Level1\_Even\_DateIn, }
\NormalTok{           Level1\_Even\_TimeIn))}
\end{Highlighting}
\end{Shaded}

\hypertarget{adjusting-wave-day}{%
\subsubsection{Adjusting Wave Day}\label{adjusting-wave-day}}

\begin{itemize}
  \item $DT_i$ combines $D_i$ and $T_i$: `DateTime` variable
  \item $DT_0$ is the first response `DateTime` for each patient
  \item $W_i$ is the adjusted `Wave\_Day` variable
  \item Add a grace period $G$ for calculating the adjusted $W_i$, in our case $G=6$ hours
  \item Calculate the datetime difference $H_i$ in **hours** from the first response, incorporating the grace period:
  $$
  H_i = DT_i-DT_{i-1}
  $$
  \item Then apply the grace period indicator $I_i$:
  $$
  I_i=
  \begin{cases}
  1 & \text{if } H_{i} \leq 24 + G \\
  \left\lceil\frac{H_{i} - G}{24}\right\rceil & \text{otherwise}
  \end{cases}
  $$
  \item The initial response for `Wave\_Day` is 1, i.e., $W_0=1$, then the adjusted `Wave\_Day` $W_i$ is 
  $$
  W_i=\sum_{i=0}^{i-1} I_i
  $$
\end{itemize}

\begin{Shaded}
\begin{Highlighting}[]
\DocumentationTok{\#\# Consecutive Days {-} Grace Period 6 hours}
\NormalTok{data\_paper }\OtherTok{\textless{}{-}}\NormalTok{ data\_paper }\SpecialCharTok{|\textgreater{}}
  \FunctionTok{mutate}\NormalTok{(}\AttributeTok{Lev1\_DateTimeIn =} \FunctionTok{as.POSIXct}\NormalTok{(}\FunctionTok{strptime}\NormalTok{(}\FunctionTok{paste}\NormalTok{(Level1\_Even\_DateIn, }
\NormalTok{                                                     Level1\_Even\_TimeIn), }
                                          \AttributeTok{format=}\StringTok{"\%Y{-}\%m{-}\%d \%H:\%M:"}\NormalTok{))) }\SpecialCharTok{|\textgreater{}}
  \FunctionTok{arrange}\NormalTok{(ID, Lev1\_DateTimeIn) }\SpecialCharTok{|\textgreater{}}
  \FunctionTok{group\_by}\NormalTok{(ID) }\SpecialCharTok{|\textgreater{}}
  \FunctionTok{mutate}\NormalTok{(}
    \AttributeTok{TimeDiffHours =} \FunctionTok{as.numeric}\NormalTok{(}\FunctionTok{difftime}\NormalTok{(Lev1\_DateTimeIn, }
                                        \FunctionTok{lag}\NormalTok{(Lev1\_DateTimeIn, }
                                            \AttributeTok{default =} \FunctionTok{first}\NormalTok{(Lev1\_DateTimeIn)), }
                                        \AttributeTok{units =} \StringTok{"hours"}\NormalTok{)), }\CommentTok{\# T diff}
    \AttributeTok{WithinGracePeriod =} \FunctionTok{if\_else}\NormalTok{(TimeDiffHours }\SpecialCharTok{\textless{}=} \DecValTok{30}\NormalTok{, }
                                \DecValTok{1}\NormalTok{, }
                                \FunctionTok{ceiling}\NormalTok{((TimeDiffHours }\SpecialCharTok{{-}} \DecValTok{6}\NormalTok{) }\SpecialCharTok{/} \DecValTok{24}\NormalTok{)), }\CommentTok{\# Check grace period}
    \AttributeTok{Wave\_Day\_Adjusted =} \FunctionTok{cumsum}\NormalTok{(WithinGracePeriod) }\CommentTok{\# Adjusted Wave\_Day}
\NormalTok{  ) }\SpecialCharTok{|\textgreater{}}
  \FunctionTok{ungroup}\NormalTok{()}

\DocumentationTok{\#\# Check if the above approach is correct}
\NormalTok{checkdf }\OtherTok{\textless{}{-}}\NormalTok{ data\_paper }\SpecialCharTok{|\textgreater{}} \FunctionTok{select}\NormalTok{(}\FunctionTok{c}\NormalTok{(ID, }
\NormalTok{                                  Lev1\_DateTimeIn, }
\NormalTok{                                  TimeDiffHours, }
\NormalTok{                                  WithinGracePeriod, }
\NormalTok{                                  Wave\_Day\_Adjusted))}
\NormalTok{checkdf2 }\OtherTok{\textless{}{-}}\NormalTok{ checkdf }\SpecialCharTok{|\textgreater{}} \FunctionTok{subset}\NormalTok{(ID }\SpecialCharTok{==} \DecValTok{2072}\NormalTok{) }\CommentTok{\# Weird ID 2072}
\FunctionTok{max}\NormalTok{(checkdf}\SpecialCharTok{$}\NormalTok{Wave\_Day\_Adjusted)}
\end{Highlighting}
\end{Shaded}

\begin{verbatim}
## [1] NA
\end{verbatim}

\begin{Shaded}
\begin{Highlighting}[]
\DocumentationTok{\#\# Fill in the gap of Wave\_Day}
\NormalTok{data\_paper2 }\OtherTok{\textless{}{-}}\NormalTok{ data\_paper }\SpecialCharTok{|\textgreater{}}
  \FunctionTok{group\_by}\NormalTok{(ID) }\SpecialCharTok{|\textgreater{}}
  \FunctionTok{complete}\NormalTok{(}\AttributeTok{Wave\_Day =} \DecValTok{1}\SpecialCharTok{:}\DecValTok{14}\NormalTok{) }\SpecialCharTok{|\textgreater{}}
  \FunctionTok{ungroup}\NormalTok{()}
\end{Highlighting}
\end{Shaded}

\begin{Shaded}
\begin{Highlighting}[]
\DocumentationTok{\#\# Check the aberrant values}
\FunctionTok{summary}\NormalTok{(data\_paper2}\SpecialCharTok{$}\NormalTok{IndexLev1\_NegativeAffect\_Total) }\CommentTok{\# Lev 1 Negative Affect }
\end{Highlighting}
\end{Shaded}

\begin{verbatim}
##    Min. 1st Qu.  Median    Mean 3rd Qu.    Max.    NA's 
##   0.000   8.667  20.500  25.154  36.667  97.333    1220
\end{verbatim}

\begin{Shaded}
\begin{Highlighting}[]
\FunctionTok{summary}\NormalTok{(data\_paper2}\SpecialCharTok{$}\NormalTok{IndexLev1\_Catastrophizing\_Total) }\CommentTok{\# Lev 1 Catas}
\end{Highlighting}
\end{Shaded}

\begin{verbatim}
##    Min. 1st Qu.  Median    Mean 3rd Qu.    Max.    NA's 
##   0.000   2.792  12.667  23.018  37.750 100.000    1220
\end{verbatim}

\begin{Shaded}
\begin{Highlighting}[]
\FunctionTok{summary}\NormalTok{(data\_paper2}\SpecialCharTok{$}\NormalTok{IndexLev2\_QST\_BaselinePPTh) }\CommentTok{\# Lev 2 PPThs?}
\end{Highlighting}
\end{Shaded}

\begin{verbatim}
##    Min. 1st Qu.  Median    Mean 3rd Qu.    Max.    NA's 
##    67.0   235.5   369.0   395.3   473.5  1200.0    1313
\end{verbatim}

\begin{Shaded}
\begin{Highlighting}[]
\FunctionTok{summary}\NormalTok{(data\_paper2}\SpecialCharTok{$}\NormalTok{IndexLev2\_QST\_TSPAve) }\CommentTok{\# Lev 2 TSP?}
\end{Highlighting}
\end{Shaded}

\begin{verbatim}
##    Min. 1st Qu.  Median    Mean 3rd Qu.    Max.    NA's 
##   -2.50    1.50    7.50   13.93   15.00   94.00    1276
\end{verbatim}

\begin{Shaded}
\begin{Highlighting}[]
\FunctionTok{summary}\NormalTok{(data\_paper2}\SpecialCharTok{$}\NormalTok{IndexLev2\_QST\_CpmTrialAve) }\CommentTok{\# Lev 2 CPM?}
\end{Highlighting}
\end{Shaded}

\begin{verbatim}
##    Min. 1st Qu.  Median    Mean 3rd Qu.    Max.    NA's 
##   37.56  103.21  119.69  122.83  135.13  251.76    1321
\end{verbatim}

\[
APE(t_i)=I(PAIN(t_i)-PAIN(t_i-1) \geq 20)
\]

\begin{Shaded}
\begin{Highlighting}[]
\DocumentationTok{\#\# APE index}
\end{Highlighting}
\end{Shaded}


\end{document}
